\section{Lineare Least-Squares Regression mit Regularisierung in Python}


\subsection{
    Betrachtung des Programmgerüst \textit{V2A1\textunderscore LinearRegression.py}
}

\subsubsection{Erklärung der Funktionen: \textit{fun\textunderscore true(), generateDataSet(), getDataError(), phi\textunderscore polynomial()}}

\textbf{fun\textunderscore true():} Berechnet die Funktionswerte der Parabel
$y(x) = w0 + w1*x + w2*x$. Die X-Werte werden als Nx1-dimensionales np.array übergeben.

\noindent
\vspace{5px}
\textbf{generateDataset():} Fügt den von \textit{fun_true()} berechneten Funktionswerten ein Rauschen hinzu. Rückgabewerte sind die X-Werte sowie die Funktionswerte T mit Rauschen.

\noindent
\vspace{5px}
\textbf{getDataError():} Berechnet die Datenfehler (Least Squares) zwischen Vorhersagedaten und den aus \textit{generateDataSet()} erzeugten \textit{True Target-Werten}.

\noindent
\vspace{5px}
\textbf{phi\textunderscore polinomial():} Generiert jeweils einen Merkmalsvektor phi\textunderscore x. In anderen Worten eine Zeile der Designmatrix PHI.

\subsubsection{Von welcher Funktion sind die Original-Daten (xn, tn) gesampelt?}

Die Python-Funktion \textit{fun\textunderscore true()} sampelt die Daten anhand der mathematischen Funktion $y(x) = w0 + w1*x + w2*x$.

\subsubsection{Wie lauten die Basisfunktionen $phi_j(x)$ für j = 1, ...,deg des linearen Modells?}

$phi_j (x) = (1 x x^2 x^3 .. x^deg)$
Die Basisfunktion kann durch die Variable \textit{deg} angepasst werden.

\subsubsection{Welche Rolle hat die Variable lmdba?}
Durch geeignete Wahl von lmbda kann man die Überanpassung vermeiden.
Dient demnach zur Regularisierung.

\subsubsection{Worin unterscheiden sich die Variablen X,T von X\\textunderscore test,T\\textunderscore test?}
Um die Qualität der ermittelten Regressions-Funktion zu prüfen, werden Trainingsdaten von Testdaten getrennt.

\subsubsection{Was stellen im Plot die grünen Kreuze/Punkte, grüne Kurve, rote Kurve dar?}
\textbf{Grüne Punkte:} stellen die generierten Testdatenpunkte dar.\\
\textbf{Grüne Kreuze:} stellen die generierten Datenpunkte dar.\\
\textbf{Grüne Kurve:} stellt die ursprüngliche Parabel-Funktion dar.\\
\textbf{Rote Kurve:} stellt die ermittelte Trennkurve dar.

\subsection{Vervollständigung des Programm}

\subsubsection{Implementierung der Berechnung der regularisierten Least-Squares-Gewichte W\textunderscore LSR als M x 1-Matrix}

\subsubsection{Implementierung der Berechnung der Prognosewerte Y als N x 1-Matrix}

\subsection{Programmtest ohne Regularisierung}

\subsubsection{Welche optimalen Gewichte WLSR erhalten Sie für Polynomgrad 5? 
                Wie groÿ ist der Lern-Datenfehler ED (WLSR)? 
                Wie groÿ ist der Fehler auf den Testdaten? 
                Warum ist der Test-Daten-Fehler gröÿer als der Lern-Daten-Fehler?}

\subsubsection{Phänomene bei niedrigem bzw. hohem Polynomgrad}


\subsection{Programmtest mit Polynomgrad 9, N = 10 Daten und Regularisierung λ ≥ 0}
