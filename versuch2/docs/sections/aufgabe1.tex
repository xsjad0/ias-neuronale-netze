\section{Lineare Least-Squares Regression mit Regularisierung in Python}


\subsection{
    Betrachtung des Programmgerüst \textit{V2A1_LinearRegression.py}
}



\subsection{
    Test der implementierten Python-Funktion
}

\subsubsection{Erklärung der Funktionen: \textit{fun_true(), generateDataSet(), getDataError(), phi_polynomial()}}

\subsubsection{Von welcher Funktion sind die Original-Daten (xn, tn) gesampelt?}

\subsubsection{Wie lauten die Basisfunktionen φj(x ) für j = 1, ...,deg des linearen Modells?}

\subsubsection{Welche Rolle hat die Variable lmdba?}

\subsubsection{Worin unterscheiden sich die Variablen X,T von X_test,T_test?}

\subsubsection{Was stellen im Plot die grünen Kreuze/Punkte, grüne Kurve, rote Kurve dar?}

\subsection{
    Vervollständigung des Programm
}

\subsubsection{Implementierung der Berechnung der regularisierten Least-Squares-Gewichte W_LSR als M x 1-Matrix}

\subsubsection{Implementierung der Berechnung der Prognosewerte Y als N x 1-Matrix}

\subsection{
    Programmtest ohne Regularisierung
}

\subsubsection{Welche optimalen Gewichte WLSR erhalten Sie für Polynomgrad 5? 
                Wie groÿ ist der Lern-Datenfehler ED (WLSR)? 
                Wie groÿ ist der Fehler auf den Testdaten? 
                Warum ist der Test-Daten-Fehler gröÿer als der Lern-Daten-Fehler?}

\subsubsection{Phänomene bei niedrigem bzw. hohem Polynomgrad}


\subsection{
    Programmtest mit Polynomgrad 9, N = 10 Daten und Regularisierung λ ≥ 0
}
